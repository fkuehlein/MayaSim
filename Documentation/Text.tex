\section{Abstract}
MayaSim is an integrated agent based, cellular automata and network simulation of the ancient Maya civilisation on the Yucatan peninsula. Agents are settlements situated on a spatial landscape. These settlements grow and interact with the surrounding forest ecosystem by using ecosystem services, causing soil erosion through logging and agriculture, and succession of abandoned areas. Thereby, the model aims at representing the nexus of climate variability, primary production, hydrology, ecosystem services, forest succession, agricultural productivity, population growth and trade network growth and stability. The Project is based on previous work of Scott Heckbert \cite{heckbert} published in Netlogo and is implemented and developed in Python. The focus of further development is mainly the implementation of more realistic approaches to modeling agents income from trade and agriculture and testing Heckberts results on structural stability and parameter sensitivity. \par
Subsequently, we want to test different hypothesis for the rise and fall of the Maya civilisation using this model.
\section{MayaSim Documentation}
\subsection{General Structure}
The model consists of a class that defines the model variables in its constructor and implements a number of routines to act upon these variables. The variables define
\begin{enumerate}
	\item The Map with its climatic conditions:
	\begin{enumerate}
		\item local temperature
		\item local elevation
		\item local soil productivity
		\item local precipitation (average)
	\end{enumerate}
	\item The ecosystem containing
	\begin{enumerate}
		\item local soil conditions
		\item local forest state (climax, regrowth or cleared land)
		\item local forest memory (only undisturbed forest can regenerate)
		\item local water and water flow
		\item local precipitation (variable)
	\end{enumerate}
	\item The society describing settlements, pioneers and the trading network
	\begin{enumerate}
		\item number of settlements
		\item settlement positions 
		\item settlement population
		\item settlement birth and death rates
		\item settlement income per capita
		\item settlement cells in influence (agricultural)
		\item settlement area of influence (used to calculate cells in influence)
		\item settlement cells in influence (used to pick cells to crop)
		\item settlement cropped cells (used for agriculture resulting in soil erosion)
		\item settlement occupied cells (cropped cells plus settlement)
		\item settlement income from agriculture
		\item settlement income from ecosystem services
		\item settlement income from trade
		\item trade network
		\item population gradient (influencing the forest)
	\end{enumerate}

\end{enumerate}
Temperature, elevation, soil productivity and average precipitation are \textbf{static variables taken from data sources}. \\
The initial number of settlements and their positions are \textbf{initial conditions}. \par
\subsection{Documentation of Routines}
The routines of the model class are the following:
\begin{enumerate}
	\item update_precipitation(self, t):\\
		uses: precipitation, time \\
		calculates: spatio-temporal precipitation \\
		This modulates the initial precipitation in a 
		24 time steps period to mimic climate variations that
		are assumed to have taken place in the period of the
		rise and fall of the Maya civilisation.
	\item get_waterflow(self):\\
		uses: spatio-temporal precipitation, land patches, elevation \\
		calculates: water flow, water on cells\\
		This places `drops' on a fraction of cells and then moves them in direction of the steepest accent to mimic water flows. The results are the amount of water that flows through cells and the amount of water that remains in cells. These are used to calculate ecosystem services and agricultural productivity.
	\item forest_evolve(self, npp):
		uses: net primary productivity, land cells, population gradient, forest state, forest memory \\
		calculates: forest state, cleared land neighbors \\
		This updates the state of the forest on all land cells by repeatedly iterating over the land cells, and updating their state depending on the state of their neighbors and their memory.\\
		Forest deteriorates with a natural probability probability that further increases if the population gradient is positive (i.e.\ if there is people around).\\
		Forest cells regenerate from cleared (state 1) to regrowth (state 2) after a certain number of steps and from regrowth (state 2) to climax (state 3) after another number of steps and under the conditions that a certain number of his neighboring cells are in climax state already. \\
		Finally, the routine updates the list of cleared land neighbors according to the cells that have regenerated.
	\item net_primary_prod(self):\\
		uses: spatio-temporal precipitation, temperature\\
		calculates: net primary productivity (npp)\\
		This routine calculates the net primary productivity according to 
		\begin{equation}
			npp = 3000 \ {\rm min}( 1 - {\rm exp}(-0.000664 \ sp), \frac{1}{1+{\rm exp}(1.315 - 0.119 \ T)}
			\label{npp}
		\end{equation}
		where $sp$ is the spatio-temporal precipitation, $T$ is the temperature and $npp$ is the net primary productivity.
	\item get_ag(self, npp, wf):\\
		uses: net primary productivity, water flow, soil condition, soil productivity, slope \\
		calculates: agricultural productivity\\
		This routine calculates the local agricultural productivity from net primary productivity, water flow, soil productivity and soil condition in a linear additive way.
	\item get_ecoserv(self, ag, wf):\\
		uses: agricultural productivity, water flow, forest state \\
		calculates: ecosystem services\\
		This routine calculates the ecosystem services of a cell from agricultural productivity, water flow and forest state in a linear additive way.
	\item benefit_cost(self, ag_In):\\
		uses: agricultural productivity, max yield, slope yield\\
		calculates: cost benefit analysis of cropping a cell\\
		The calculation has the form:
		\begin{equation}
			cb = my ( 1 - os \ {\rm exp}(sy \ ag) )
			\label{benefit_cost_analysis}
		\end{equation}
		where $cb$ is the net benefit of cropping a cell, $my$ is the maximum yield, $sy$ is the slope yield and $os$ is the origin shift.
		(these quantities sure need some explanation..)
	\item get_cells_in_influence(self):\\
		uses: population, settlement positions\\
		calculates: settlement area of influence, settlement cells in influence\\
		This routine first calculates the area of influence for all settlements according to
		\begin{equation}
			ai = \frac{P^{0.8}}{60}
			\label{are_of_influence}
		\end{equation}
		where $ai$ is the area of influence of a settlement with population $P$.\\
		Then the cells under influence of the settlement are all those, that are within its area of influence.
	\item get_cropped_cells(self, bca): \\
		uses: bca, settlement cropped cells, settlement population, area per cell, settlement coordinates, min people per cropped cell, max people per cropped cell\\
		calculates: newly cropped cells, abandoned cells, settlement age, settlement cropped cells \\
		First, the routine calculates the 'agricultural population density' $apd$ for each settlement:
		\begin{equation}
			apd = \frac{P}{N_{cropped cells} A_{cell}}
			\label{apd}
		\end{equation}
		from the settlements population $P$, number of cropped cells $N_{cropped cells}$ and are per cell $A_{cell}$. \\
		Second, the routine calculates a utility $U_{crop}$ for each cell from its benefit cost analysis $bca$, establishment cost $c_{est}$, distance from the settlement $d$, population of the settlement $P$ and a travel cost distance $c_{travel}$:
		\begin{equation}
			U_{crop} = bca - c_{est} - \frac{c_{travel} d}{\sqrt{P}}
			\label{U_crop}
		\end{equation}
		Third, the cropped cells for each settlement are updated in three steps: \\
		\textbf{A)} if $apd$ is lower then 'min people per cropped cell', a number of $30/apd$ cells is abandoned beginning with the ones with lowest utility first. \\
		\textbf{B)} all cells with negative utility are abandoned. \\
		\textbf{C)} if $apd$ is higher then 'max people per cropped cell', a number of $apd$/'max people per cropped cell' are newly cropped. Newly cropped cells are chose amongst the cells under influence of the settlement with the ones with highest utility first. \\
		Finally, the list of cropped cells is updated.
	\item get_pop_mig(self):\\
		uses: settlement population, min death rate, max death rate, min birth rate, max birth rate, min, migration rate, max migration rate, establishment cost \\
		calculates: birth rate, death rate, migration rate, settlement population, settlement emigration pressure \\
		The death rate is limited by min and max values $\delta_{min}$ and $\delta_{max}$ and is linearly depending on the income per capita $I$ in between:
		\begin{equation}
			\delta = \delta_{max} - (\delta_{max} - \delta_{min})I
			\label{deat_rate}
		\end{equation}
		Optionally, the birth rate decreases with the settlement population $P$ for populations greater 5000 according to:
		\begin{equation}
			b = s - (b_{max} - b_{min}) \frac{P}{1000}.
			\label{birth_rate}
		\end{equation}
		The population is then changed according to 
		\begin{equation}
			P(t+\Delta t) = P(t) (b - \delta).
			\label{population_growth}
		\end{equation}
		With the updated population values, the routine calculates the outward migration pressure for the settlements.
		Similarly to the death rate of the settlements, the migration rate scales between a minimum value $m_{min}$ and a maximum values $m_{max}$ linearly with the real income per capita $I$:
		\begin{equation}
			m = m_{max} - (m_{max} - m_{min}) I.
			\label{migration_rate}
		\end{equation}
		The migration pressure is given by the migration rate times the settlement population.
	\item update_pop_gradient(self):\\
		uses: settlement population, settlement coordinates, cell coordinates\\
		calculates: population gradient for each cell \\
		The population gradient $\pi_j$ on cell $j$ is calculated from the population of the settlements $P_i$, their position $x_i$ and the cells position $x_j$ as:
		\begin{equation}
			\pi_j = \sum_i \frac{P_i}{300(1 + |x_i - x_j|)}
			\label{population_gradient}
		\end{equation}
		where $|*|$ is the euclidean distance between $x_i$ and $x_j$.
		
	\item evolve_soil_ged(self)\\
		uses: settlements cropped cells, local soil conditions\\
		calculates: local soil conditions \\
		local soil conditions decrease with the soil degeneration rate if a cell is cropped and increase with the soil regeneration rate if the cell is occupied by climax (stage 3) forest.
	\item get_rank(self):\\
		uses: settlement population\\
		calculates: settlement ranks\\
		The routine assigns a settlement rank 1 if its population exceeds 4000, rank 2 if its population exceeds 7000 and rank 3 if its population exceeds 9500.
	\item build_routes(self):\\
		uses: settlement ranks, settlement positions, settlements adjacency matrix\\
		calculates: changes in settlements adjacency matrix\\
		If settlements have a rank higher then the number of trade links they uphold, they create a new link. New links can be established to ` nearby' cities where nearby means a distance lower then:
		\begin{equation}
			d_{max} = 31 \left( \frac{t_i}{2 t_3} +1 \right)
			\label{closenessk_threshold}
		\end{equation}
		where $t_i$ is the population threshold matching the settlements rank. Of all the 'nearby' settlements, the trade link is established with the one with the largest population.
	\item get_comps(self):\\
		uses: settlement adjacency matrix\\
		calculates: settlements degree and component size in the trade network\\
	\item get_centrality(self):\\
		uses: settlement adjacency matrix\\
		calculates: settlements centrality in the trade network\\
	\item get_crop_income(self, bca):\\
		uses: agricultural benefit cost analysis, settlements cropped cells\\
		calculates: settlements crop yield\\
		There is two ways that the yield of settlements' crops is calculated (although only one makes sense..)
		The crop yield is either calculated from the sum of the benefit cost analysis of all cropped cells of the settlement:
		\begin{equation}
			Y_{i, ag} = r_{bca}\sum_j (bca_{i,j})
			\label{crop_yield}
		\end{equation}
		where $bca_{i,j}$ is the benefit cost calculation for all cells $j$ of settlement $i$ and $r_bca$ is a weighting factor. \\
		Alternatively (originally but mistakenly, as I'd say) the crop yield was calculated from the mean of the benefit cost analysis of all cropped cells of the settlement:
		\begin{equation}
			Y_{i, ag} = r_{bca}{\rm mean}(bca_{i, j}).
			\label{crop_yield_mean}
		\end{equation}
	\item get_eco_income(self, es):\\
		uses: ecosystem services
		calculates: settlements income from ecosystem services\\
		The routine calculates settlements income from ecosystem services $I_{i, es}$ from the mean of all cells $j$ under influence by the settlement:
		\begin{equation}
			I_{i, es} = r_{es}{\rm mean}(es_{i,j})
			\label{ecosystem_income}
		\end{equation}
		This is (sames as with the income from agriculture) pretty wrong and should be changed in my opinion.
	\item get_trade_income(self): \\
		uses: settlements component size, settlements centrality, $r_{trade}$\\
		calculates: settlements trade income\\
		The routine calculates the settlements trade income $I_{i,trade}$ from settlements component size $c_i$ and centrality $z_i$ according to:
		\begin{equation}
			\frac{1}{30}\left( 1 + \frac{c_i}{z_i} \right)^{0.9}
			\label{trade_income}
		\end{equation}
		There is no source or line of argument for this way of calculating income from trade.
	\item get_real_income(self):\\
		uses: settlements income from trade, settlements income from ecosystems, settlements income from agriculture\\
		calculates: settlements total real income per capita\\
		\begin{equation}
			I_i = \frac{I_{i, trade}+I_{i, es} + I_{i, ag}}{P_i}
			\label{real_income_pc}
		\end{equation}
	\item migration(self, es):\\
		uses: ecosystem services, settlements cells in influence, settlements migration pressure, settlement coordinates, \\
		calculates: positions of newly spawned cities\\
		The routine first calculates the set $G$ of land cells $A$ that is not under the influence $C$ of a city:
		\begin{equation}
			G = A \setminus C
			\label{vacant_cells}
		\end{equation}
		Then, if this set is at least 75 cells large, if the outward migration pressure (the number of people willing to emigrate) is larger then 400 and with a probability $p>05$ a set $M$ of 75 cells is randomly drawn from the set of vacant cells $G$. The migration utility of these cells is calculated from their ecosystem services $es_i$ and distance from settlement of origin $d_ij$:
		\begin{equation}
			U_{i,m} = u_{es} es_i + u_{t}d_i
			\label{migration_utility}
		\end{equation}
		The location for the new settlement is then the cell of the set $M$ with the highest utility $U_{i,m}$.
		If this location has no neighbors that are closer then $7.5$ in units of the model, a new city is spawned there.
	\item kill_cities(self):\\
		This routine removes settlements that have zero population or zero cropped cells, if their age>5.
		It removes their entries from all model variables and decreases the number of settlements by the appropriate number.
	\item spawn_city(self, a, b, mig_pop)\\
		uses: new settlement location, new settlement population\\
		This routine adds a new settlement to the model. It appends values to all system variables accordingly.
	\item run(self, t_max, location)
		This routine runs the model by running a loop for the given number of time steps and calling the necessary routines (described above) in the following order:
		\begin{enumerate}
			\item update_precipitation()
			\item net_primary_prod()
			\item forest_evolve(npp)
			\item get_waterflow()
			\item get_ag()
			\item get_ecoserv()
			\item benefit_cost()
			\item get_cells_in_influence()
			\item get_cropped_cells()
			\item get_crop_income()
			\item get_eco_income()
			\item evolve_soil_deg()
			\item update_pop_gradient()
			\item get_rank()
			\item build_routes()
			\item get_comps()
			\item get_centrality()
			\item get_trade_income()
			\item get_real_income()
			\item get_pop_mig()
			\item migration()
			\item kill_cities()
		\end{enumerate}
\end{enumerate}
\subsection{Algorithm Flowchart}
The following flowchart depicts the dependency of the previously named functions via their respective outputs:
\begin{figure}[H]
	\centering
	\includegraphics[width = \textwidth]{Flowchart.pdf}
	\caption{This flowchart depicts the algorithm structure and the dependency of the functions via their respective outputs.}
	\label{fig:flowchart}
\end{figure}
\subsection{Model Results}
Show some model results as stylized facts of overshoot and collapse of
population. \\
Also compare results from runs with income from agriculture and ecosystem
services calculated as the sum or average over the cells in the settlements
influence.
\subsection{Perspectives}
We want to use the existing model to test different possible triggers for the
rise and fall of the Maya civilisation on the Yucatan peninsula. The original
hypothesis was: A prolonged drought led to sudden decline of agricultural
income and consequentially decline in population and collapse of trade
networks.\\
Alternative hypotheses are hypotheses are:
\begin{itemize}
    \item[1] The trade networks are no necessary condition for the collapse. The
        only reason is the decline in income from agriculture and ecosystem
        services.
    \item[2] A drought event is not necessary for the collapse. The collapse is
        caused by the overshoot and collapse in the predator prey dynamics
        between population and forest.
\end{itemize}

\section{Working Paper}
There are two interlinked strands of work in this project. One strand is concerned with \textbf{improvements of the model} whereas the other strand is concerned with the study of \textbf{persistence and stability of complex society states} in the model.

\subsection{First Branch: Persistence and stability of Complex Society States}
There focus of the first branch in this project is given by the scope of an abstract, that was handed in to the RESILIENCE 2017 conference in Stockholm, Sweden.
It reads:\par
\textbf{Abstract} \par
The rise and fall of the ancient Maya society is debated as an iconic example for the collapse or catastrophic reorganization of a complex social-ecological system due to climate variability.
Building on previous work by Scott Heckbert et al. on the MayaSim model, we use an integrated agent-based, cellular-automata and network-model to investigate this particular case of social-environmental coevolution. The model is able to produce classical overshoot and collapse scenarios as known from the Polynesian islands as well as a stable "complex society" state depending on choice of parameters.
We describe the persistence resilience of the regionally organized "complex society" state to drought events using a basin stability approach and analyze its structural stability to e.g. the population dynamics of the system. We also examine its possible tipping and degenerational transformation to a lesser organized, yet more resilient subsistence state of the system.\par

Therefore, the first goal is to tweak the system to show a stable complex society state. \par 

\subsection{Second Branch: Model Improvements}
There is a Number of things that can be done in this direction.
\begin{itemize}
	\item In the original Model, the income of a city is calculated as the mean of the income from the cells in its influence. This is difficult to justify. It would make much more sense, to calculate income as the sum of the income from the cells in influence.
	\item In the original model, cities are removed from the simulation, if they do not have agriculturally cultivated cells after an initial period of five years. Yet, the birth and death rates only depend on income per capita. It could be justified to remove settlements that have less then a minimum income per capita or that have less then a minimum number of inhabitants. But removing cities without agriculture with no direct prior feedback from agriculture to demographic dynamics does not make much sense.
	\item Geologic studies have shown, that soil degradation in the ancient Maya civilization depended on the stage of development of the respective settlement. Settlements in the process of their foundation and in early stages of development caused high soil degradation, whereas fully developed cities caused only very little soil development due to advanced processes and techniques in their agricultural activities. This is not yet represented in the Model.
	\item The income from trade in the model is only based on the settlements position in the trade network. This does not make much sense and should be changed in the future.
\end{itemize}

\section{Progress Report}
Progress on both branches is reported in the following.
\subsection{Improvement of the calculation of income from agriculture and ecosystem services}
To evaluate any changes in the model, I need a reference run of the original setup, that I can compare the changes to. Therefore, I will first present this reference run here.


\subsection{Human Inflicted Soil Erosion}

\subsection{Prerequisites for the Persistence of a Complex Society State}

My hypothesis is that if settlements don't die if they don't have any agriculturally cultivated cells anymore and if the possible income from trade is high enough to sustain their trading relationships, the complex society will be able to sustain itself.